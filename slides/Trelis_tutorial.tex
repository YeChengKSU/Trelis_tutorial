\documentclass[fleqn]{beamer}
\usepackage[british]{babel}
\usepackage{graphicx,ru,url}
\usepackage{amsmath}
% Use Times for math font and text font.
\RequirePackage[T1]{fontenc}
%\RequirePackage{txfonts}
% bold math must be loaded after Times font
\usepackage{bm}
\usepackage{booktabs} % nice rules (thick lines) for tables
\usepackage{microtype} % improves typography for PDF
\usepackage{xcolor} % Allows colors in fonts
\usepackage{tikz} % Allows creation of tikz pictures
\usepackage{verbatim}
\usetikzlibrary{arrows,shapes,snakes}
\usepackage{hyperref}
% \usepackage{minted} % Allows printing of python (or other) code
% \newminted{python}{fontsize=\scriptsize, 
%     linenos,
%     numbersep=8pt,
%     gobble=4,
%     frame=lines,
%     bgcolor=bg,
%     framesep=3mm}

% The title of the presentation:
%  - first a short version which is visible at the bottom of each slide;
%  - second the full title shown on the title slide;
\title[Trelis Tutorial]{
    Trelis Tutorial}

% Optional: a subtitle to be displayed on the title slide
%\subtitle{Show where you're from}

% The author(s) of the presentation:
%  - again first a short version to be displayed at the bottom;
%  - next the full list of authors, which may include contact information;
\author[Ye Cheng]{
    Ye Cheng}

% The institute:
%  - to start the name of the university as displayed on the top of each slide
%    this can be adjusted such that you can also create a Dutch version
%  - next the institute information as displayed on the title slide
\institute[Kansas State University]{
    Mechanical and Nuclear Engineering \\
    Kansas State University}

% Add a date and possibly the name of the event to the slides
%  - again first a short version to be shown at the bottom of each slide
%  - second the full date and event name for the title slide
\date[CORPS-G]{
    CORPS-G Group Meeting \\
    02/22/2019}

\begin{document}
    % These two commands allow bonus slides at the end
    % The bonus slides will not be numbered
    \newcommand{\beginbackup}{
        \newcounter{framenumbervorappendix}
        \setcounter{framenumbervorappendix}{\value{framenumber}}
    }
    \newcommand{\backupend}{
        \addtocounter{framenumbervorappendix}{-\value{framenumber}}
        \addtocounter{framenumber}{\value{framenumbervorappendix}} 
    }
    
    \begin{frame}
        \titlepage
    \end{frame}
%%%%%%%%%%%%%%%%%%%%%%%%%%%%%%%%%%%%%%%%%%%%%%%%%%%%%%%%%%%%%%%%%%%%%%%%%%%  
% Outline
    \begin{frame}
        \frametitle{Outline}
        \begin{block}{Presentation Outline}
            \begin{itemize}
                \item Installation
                \begin{itemize}
                    \item Trelis installation
                    \item MCNP plugin installation
                \end{itemize}
                \item Simple meshing instruction
                \item Decomposition
                \item ITEM tool
                \item examples
		  \begin{itemize}
		   \item single pin element meshing
		   \item UWNR bare core meshing
		  \end{itemize}

            \end{itemize}
        \end{block}
    \end{frame}
%%%%%%%%%%%%%%%%%%%%%%%%%%%%%%%%%%%%%%%%%%%%%%%%%%%%%%%%%%%%%%%%%%%%%%%%%%%    
\section{Installation}
\begin{frame}
 \frametitle{Install Trelis}
 \begin{itemize}
 \item Request Trelis Trial or license on \textbf{www.csimsoft.com}.
 \item The latest Trelis version is 16.3.5.
 \item Install the Trelis package on your machine.
  \end{itemize}
\end{frame}

\begin{frame}
\frametitle{Install Trelis mcnp-plugin}
\begin{block}{What is Trelis mcnp-plugin}
\begin{itemize}
 \item MCNP2CAD  is a program developed by CNERG from University of Wisconsin to extract the geomtery from MCNP input files and write it out
to a Trelis/Cubit CAD file.
 \item The workflow is: MCNP6 script $\rightarrow$ Trelis MCNP2CAD plugin $\rightarrow$ Trelis CAD file $\rightarrow$ Trelis meshing $\rightarrow$ Further transport code model(Proteus, Rattlesnake, Mammoth, etc). 
\end{itemize}
\end{block}
\end{frame}

\begin{frame}
\frametitle{Install Trelis mcnp-plugin(Cont.)}
\begin{itemize}
\item Download from MCNP2CAD Github repo: \textbf{https://github.com/svalinn/mcnp2cad}. Alternatively, follow the instruction in \textbf{installation\_guide/Trelis\_installation\_instruction}.
 \item There could be package dependency problems causing the mcnp-plugin not able to be loaded. Typically there should be two:
 \begin{itemize}
  \item  \textbf{libMOAB.so.0} is not found because that the \textbf{LD\_LIBRARY\_PATH} has not been set in new user's \textbf{bashrc} file
  \item textbf{libarmadillo.so.6} is not found because the package \textbf{libarmadillo-dev}, on which Trelis-plugins have a dependence, has not been
   installed on new user's computer
 \end{itemize}

The detailed Trelis-fixing instruction can be found in \textbf{installation\_guide/fixing\_mcnp2cad}.
\end{itemize}
\end{frame}
%%%%%%%%%%%%%%%%%%%%%%%%%%%%%%%%%%%%%%%%%%%%%%%%%%%%%%%%%%%%%%%%%%%%%%%%%%%  
\section{Simple meshing instruction}
\begin{frame}
        \frametitle{Trelis GUI window}
        \begin{center}
            \includegraphics[trim=.1cm .25cm 2.0cm .4cm, clip=true, totalheight=.8\textheight]{figures/gui_window.png}
        \end{center}
\end{frame}

\begin{frame}
\frametitle{Create a simple mesh}
\begin{itemize}
 \item Creating the geometry
 \item Setting the interval sizes and meshing schemes
 \item Meshing the geometry
 \item Specifying the boundary conditions
 \item Exporting the mesh
\end{itemize}
\end{frame}

\begin{frame}
 \frametitle{Generated mesh for brick with cylindrical Hole}
 \begin{itemize}
  \item Step 1: Beginning Execution
  \item Step 2: Creating the Brick
  \item Step 3: Creating the Cylinder
  \item Step 4: Adjusting the Graphics Display
  \item Step 5: Forming the Hole
  \item Step 6: Setting Interval Sizes
  \item Step 7: Surface Meshing
  \item Step 8: Volume Meshing
  \item Step 9: Inspecting the Model
  \item Step 10: Materials and Boundary Conditions
  \item Step 11: Exporting the Mesh
 \end{itemize}
\end{frame}

\begin{frame}
 \frametitle{Generated mesh for brick with cylindrical Hole(Cont.)}

\begin{columns}[t]
\column{.33\textwidth}
\centering
\includegraphics[width=3cm,height=2cm]{figures/step_2.png}\\
\includegraphics[width=3cm,height=2cm]{figures/step_3.png}\\
\includegraphics[width=3cm,height=2cm]{figures/step_4.png}
\column{.33\textwidth}
\centering
\includegraphics[width=3cm,height=2cm]{figures/step_5.png}\\
\includegraphics[width=3cm,height=2cm]{figures/step_7.png}\\
\includegraphics[width=3cm,height=2cm]{figures/step_8.png}\\
\column{.33\textwidth}
\centering
\includegraphics[width=3cm,height=2cm]{figures/step_9.png}\\
\end{columns}

\end{frame}
%%%%%%%%%%%%%%%%%%%%%%%%%%%%%%%%%%%%%%%%%%%%%%%%%%%%%%%%%%%%%%%%%%%%%%%%%%%
\section{Decomposition}
\begin{frame}
\frametitle{sweepable model}
\begin{block}{What volume is sweepable}
 Sweepable volumes can be comprised of many
different topologies. We typically classify sweeping problems into three groups, based on the number of
source/target surfaces.
\begin{itemize}
 \item One-to-one: A volume with a one source surface and one target surface.
 \item Many-to-one: A volume with multiple source surfaces and one target surface
 \item Multisweep (or Many-to-Many): A volume with multiple target surfaces
\end{itemize}
\end{block}
\end{frame}

\begin{frame}
 \frametitle{sweepable model(cont.)}
 \begin{columns}[t]
\column{.33\textwidth}
\centering
\includegraphics[width=4cm,height=3cm]{figures/one_to_one.png}\\{One to one}
\column{.33\textwidth}
\centering
\includegraphics[width=4cm,height=3cm]{figures/many_to_one.png}\\{Many to one}
\column{.33\textwidth}
\centering
\includegraphics[width=4cm,height=3cm]{figures/many_to_many.png}\\{Many to many}
\end{columns}
\end{frame}

\begin{frame}
\frametitle{General steps of decomposition}
\begin{itemize}
\item Webcut
\item Imprint \& Merge
\item Select meshing strategy 
\item Set the interval size
\item Select your sweet path
\item Use as few webcuts as possible
\item Set the source and target surfaces
\item View meshing quality   
\end{itemize}
\end{frame}



  \begin{frame}
 \frametitle{Decomposition example(cont.)}
 \begin{columns}[t]
 \column{.5\textwidth}
 \centering
 \includegraphics[width=5cm,height=4cm]{figures/decomp_exterior.png}\\{Exterior look}
 \column{.33\textwidth}
 \centering
 \includegraphics[width=5cm,height=4cm]{figures/decomp_interior.png}\\{Interior lookz}
 \end{columns}
 \end{frame}
 
  \begin{frame}
 \frametitle{Decomposition example(cont.)}
 \centering{Source}
 \begin{columns}[t]
 \column{.2\textwidth}
 \centering
 \includegraphics[width=2.5cm,height=1.5cm]{figures/s1.png}\\
 \includegraphics[width=2.5cm,height=1.5cm]{figures/t1.png}\\
  \column{.2\textwidth}
 \centering
 \includegraphics[width=2.5cm,height=1.5cm]{figures/s2.png}\\
 \includegraphics[width=2.5cm,height=1.5cm]{figures/t2.png}\\
  \column{.2\textwidth}
 \centering
 \includegraphics[width=2.5cm,height=1.5cm]{figures/s3.png}\\
 \includegraphics[width=2.5cm,height=1.5cm]{figures/t3.png}\\
  \column{.2\textwidth}
 \centering
 \includegraphics[width=2.5cm,height=1.5cm]{figures/s4.png}\\
 \includegraphics[width=2.5cm,height=1.5cm]{figures/t4.png}\\
  \column{.2\textwidth}
 \centering
 \includegraphics[width=2.5cm,height=1.5cm]{figures/s5.png}\\
 \includegraphics[width=2.5cm,height=1.5cm]{figures/t5.png}\\
 \end{columns}
 \centering{Target}
 \end{frame} 

%%%%%%%%%%%%%%%%%%%%%%%%%%%%%%%%%%%%%%%%%%%%%%%%%%%%%%%%%%%%%%%%%%%%%%%%%%%
\section{Example}
\begin{frame}
 \frametitle {UWNR single pin model}
 \begin{itemize}
  \item Load from MCNP file
  \item Remove redundant volumes
  \item Try to identify sweepable volumes and unsweepable volumes
  \item Decomposing the volumes
  \item Mesh the single pin model
  \item export as Exodus file for further usage.
 \end{itemize}
 \end{frame}
 
  \begin{frame}
 \frametitle{UWNR single pin model(cont.)}
 \begin{columns}[t]
 \column{.5\textwidth}
 \centering
 \includegraphics[width=5cm,height=4cm]{figures/single_element_meshing.png}\\{Single element meshing}
 \column{.5\textwidth}
 \centering
 \includegraphics[width=5cm,height=4cm]{figures/single_element_mesh_quality.png}\\{Meshing quality}
 \end{columns}
 \end{frame}

 \begin{frame}
 \frametitle{UWNR bare core}
 \begin{columns}[t]
 \column{.5\textwidth}
 \centering
 \includegraphics[width=5cm,height=4cm]{figures/bare_core_mesh.png}\\{Single element meshing}
 \column{.5\textwidth}
 \centering
 \includegraphics[width=5cm,height=4cm]{figures/bare_core_mesh_quality.png}\\{Meshing quality}
 \end{columns}
 \end{frame}

\end{document}
